\documentclass[]{spie}  %>>> use for US letter paper
%\documentclass[a4paper]{spie}  %>>> use this instead for A4 paper
%\documentclass[nocompress]{spie}  %>>> to avoid compression of citations

\renewcommand{\baselinestretch}{1.0} % Change to 1.65 for double spacing
 
\usepackage{amsmath,amsfonts,amssymb}
\usepackage{graphicx}
\usepackage[colorlinks=true, allcolors=blue]{hyperref}
% my packages and commands
\usepackage{booktabs} % for nice tables
\usepackage{verbatim}
\graphicspath{{figs/}}

\title{How to: Grammarly integration with the TeXstudio}

\author[a]{Pawel Kudela}

\affil[a]{Institute of Fluid-Flow Machinery, Polish Academy of Sciences, Fiszera 14 St, 80-231 Gdansk, Poland}

\authorinfo{Further author information: (Send correspondence to Pawel Kudela)\\Pawel Kudela: E-mail: pk@imp.gda.pl, Telephone: +48 58 5225 251}

% Option to view page numbers
\pagestyle{empty} % change to \pagestyle{plain} for page numbers   
\setcounter{page}{301} % Set start page numbering at e.g. 301
 
\begin{document} 
\maketitle

\begin{abstract}
This is manual for making life easier in the Texstudio environment by the integration of Grammarly into the workflow. 
It contains some hacks in order to display a web page with a file you are currently working on. 
Grammarly will check the file for you on the web page. 
\end{abstract}

% Include a list of keywords after the abstract 
\keywords{TeXstudio, LaTeX, Grammarly }

\section{PROBLEM STATEMENT}
\label{sec:intro}  % \label{} allows reference to this section
Grammarly is a quite good piece of software for grammar checks and corrections. 
It integrates well with Microsoft Word and Google Chrome. 
However, such a tool cannot be used directly with TeXstudio or other text editors. Currently, there are no such solutions available. 
There is an issue \# 528 on Github in relation to TeXstudio development as a feature-request for Grammarly integration: \url{https://github.com/texstudio-org/texstudio/issues/528}.
However, developers don't see how it can be possible.

\section{How does it work?}

A custom command is added to be executed after the compilation of the LaTex file.
This custom command runs a batch script that runs a program for stripping out LaTeX commands. A new file without LaTeX command is saved as \verb|detex-out.tex|. 
The content of the file is loaded to the locally hosted web page. 
Local hosting is done through \href{https://chrome.google.com/webstore/detail/web-server-for-chrome/ofhbbkphhbklhfoeikjpcbhemlocgigb}{Web server for Chrome}. Grammarly will check your text because it is placed in an editable field. 
However, it is not designed in the editions in mind - Java script refreshes this web page every 2 seconds. 
Once you see some mistakes in red on the web page in the Chrome browser you can immediately correct them in the TeXstudio editor and recompile the document. 

\section{How to do it for my TeXstudio?}
You need to take several steps to do it. You can do these steps in any order but this one is preferable. I assume that you have already installed \href{https://www.grammarly.com}{Grammarly}.
\paragraph{Step 1} Copy  the \verb|make_detex.bat| batch file. 

It should be copied to the folder in which you have your \verb|.tex| document that you are currently working on.
The idea behind this batch file is to have multiple copies of it located in each folder containing the main LaTeX document, e.g.\ your conference paper, journal paper, report, etc.
Therefore, it is possible to customised it separately for each document.

\paragraph{Step 2} Setup paths in the \verb|make_detex.bat|. 

Right-click on the file and click edit.
The content of the file looks like this:
 \begin{verbatim}
 SETLOCAL
 set original="main.tex"
 set output="E:\work\Grammarly-LaTeX-man\grammarly_web\detex-out.txt"
 ::set output="\\tsclient\E\grammarly_web\detex-out.txt"
 set detex="E:\work\Grammarly-LaTeX-man\opendetex\detex.exe"
 ::set detex=%~dp0\..\..\..\bin\external\opendetex\detex.exe
 %detex% -l %original%>%output%
 ENDLOCAL
 \end{verbatim}
If the file name you are working on is other than \verb|main.tex| you need to replace it after \verb|set original=|
Caution is needed here: a complex path or file name containing white spaces should be given within quotation marks.

The second path points out to the output file which will be displayed on the web page. The name of the file is fixed as \verb|detex-out.txt|
It is already in the \verb|\grammarly_web| folder on a hard drive provided along with this manual so you only need to change the path accordingly.
The next line is actually commented out by using a double colon and it is an example of a path to the remote location.

The third path points out to the \verb|detex.exe| which is the program for stripping out LaTeX commands. This program resides in the \verb|opendetex| folder and you should also modify it accordingly. It is also within quotation marks.
The next line is also commented line. 
It is an example of a relative path. Hence, if you do not want to set up the relative path you can skip it.

\paragraph{Step 3} Install \href{https://chrome.google.com/webstore/detail/web-server-for-chrome/ofhbbkphhbklhfoeikjpcbhemlocgigb}{Web server for Chrome}. 

Once you have installed the Web Server it should be visible at \verb|chrome://apps/| in Chrome browser search bar. Click it and click the button "choose folder". It should point out to \verb|\grammarly_web| folder on a hard drive which is provided with this manual along with other files. 
The path must be compatible with the output path provided in \verb|make_detex.bat| batch script. 

Please make sure that the box "Automatically show index.html" is selected. 
For security reasons I recommend to un-tick "Accessible on local network".  

Click on the Web Server URL given below (it should be something like \verb|http://127.0.0.1:8887|). 
The web page with the text "detex-out" should show up. 
If not, just click "index.html".
This is the web page where your LaTeX document with Grammarly corrections will appear.

\paragraph{Step 4} Configure TeXstudio.

Two options are possible here: manual or automatic. 
The manual option is recommended if you want to keep you customized settings of TeXstudio. By using the automatic option you just load a pre-made profile with every setting including command highlights, editor fonts, etc.

\clearpage
\noindent \textbf{\textit{Manual option}}
Go to TeXstudio Options \(\rightarrow\) Configure \(\rightarrow\)  Build

Replace "Build and View" line by:
\begin{verbatim}
 txs:///pdflatex | txs:///view-pdf-internal --embedded| txs:///user3
 \end{verbatim}
 Add user command:
 \begin{verbatim}
 user3:detex  ?a)make_detex.bat
 \end{verbatim}
 
\noindent \textbf{\textit{Automatic option}}

Go to TeXstudio Options \(\rightarrow\) Load profile
Point out to the file \verb|detex.txsprofile| provided with this manual.

\paragraph{Enjoy!} That is it! 

After compiling your LaTeX document it should appear on the web page in the Chrome browser.

\section{Conclusions}
The presented approach enables the improvement of the workflow of writing documents in LaTeX. 
Grammarly hacked integration works well in a concert with spell checking embedded to the TeXstudio in the form of Language Tool.

\textbf{There are no more excuses to use other WYSIWYG text editor due to a lack of Grammarly integration!}

\end{document} 
