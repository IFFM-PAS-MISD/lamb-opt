 %DIF PREAMBLE START
\RequirePackage{xcolor}
%\definecolor{red}{rgb}{1,0,0}
%\definecolor{mydarkmagenta}{rgb}{0,0,0.55}
%\definecolor{blue}{rgb}{0,0,1}
%\definecolor{mycitecolor}{rgb}{0.6,0.2,0.8} %DIF PREAMBLE % dark orchid
%\definecolor{mycitecolor}{rgb}{1,0.55,0} %DIF PREAMBLE % darkorange
\definecolor{mydelcolor}{rgb}{0.464,0.531,0.598} % grey color for deleted text
\definecolor{myaddedcolor}{rgb}{0.55, 0.0, 0.55} % magenta color for added text
\RequirePackage{graphicx}%DIF PREAMBLE
\RequirePackage{tikz} %DIF PREAMBLE
\RequirePackage[normalem]{ulem} %DIF PREAMBLE

\providecommand{\DIFaddtex}[1]{{\protect\color{myaddedcolor}#1}} %DIF PREAMBLE
\providecommand{\DIFdeltex}[1]{{\protect\color{mydelcolor}\sout{#1}}}    %DIF 
\providecommand{\DIFaddcite}[1]{{\protect\color{myaddedcolor}\hypersetup{citecolor = myaddedcolor}#1}} 
\providecommand{\DIFdelcite}[1]{{\protect\color{mydelcolor}\hypersetup{citecolor = mydelcolor}#1}}    %DIF PREAMBLE
\providecommand{\DIFdelmath}[1]{{\protect\color{mydelcolor}#1}}    %DIF PREAMBLE
\providecommand{\DIFaddmath}[1]{{\protect\color{myaddedcolor}#1}}  %DIF PREAMBLE
\providecommand{\DIFdeltab}[1]{{\protect\color{mydelcolor}#1}}    %DIF PREAMBLE
\providecommand{\DIFaddtab}[1]{{\protect\color{myaddedcolor}#1}}  %DIF PREAMBLE

\providecommand{\DIFadd}[1]{\texorpdfstring{\DIFaddtex{#1}}{#1}} %DIF PREAMBLE
\providecommand{\DIFdel}[1]{\texorpdfstring{\DIFdeltex{#1}}{}} %DIF PREAMBLE
 %DIF PREAMBLE
\providecommand{\DIFdelfig}[1]{ \begin{tikzpicture}
	\node[anchor=south west,inner sep=0] (image) at (0,0) {\protect\color{mydelcolor}
		#1
	};
	\begin{scope}[x={(image.south east)},y={(image.north west)}]
	\draw[mydelcolor,line width=1 mm] (0, 1)--(1, 0); % cross line 1
	\draw[mydelcolor,line width=1 mm] (0, 0)--(1, 1); % cross line 2
	\draw[mydelcolor,line width=1 mm] (0, 0)--(0, 1)--(1,1)--(1,0)--(0,0)--(0,1); % box
	\end{scope}
	\end{tikzpicture} }
 %DIF PREAMBLE
\providecommand{\DIFaddfig}[1]{ \begin{tikzpicture}
	\node[anchor=south west,inner sep=0] (image) at (0,0) {\protect\color{blue}
		#1
	};
	\begin{scope}[x={(image.south east)},y={(image.north west)}]
	\draw[myaddedcolor,line width=1 mm] (0, 0)--(0, 1)--(1,1)--(1,0)--(0,0)--(0,1); % box
	\end{scope}
	\end{tikzpicture} }
 %DIF PREAMBLE END

\documentclass[]{spie}  %>>> use for US letter paper
%\documentclass[a4paper]{spie}  %>>> use this instead for A4 paper
%\documentclass[nocompress]{spie}  %>>> to avoid compression of citations

\renewcommand{\baselinestretch}{1.0} % Change to 1.65 for double spacing
 
\usepackage{amsmath,amsfonts,amssymb}
\usepackage{graphicx}
\usepackage[many]{tcolorbox}
\usepackage{tikz}
\usepackage[colorlinks=true, allcolors=blue]{hyperref}
% my packages and commands
\usepackage{booktabs} % for nice tables
% matrix command 
\newcommand{\matr}[1]{\mathbf{#1}} % bold upright (Elsevier, Springer)
% vector command 
\newcommand{\vect}[1]{\mathbf{#1}} % bold upright (Elsevier, Springer)
\newcommand{\ud}{\mathrm{d}}
\renewcommand{\vec}[1]{\mathbf{#1}}
\newcommand{\veca}[2]{\mathbf{#1}{#2}}
\newcommand{\bm}[1]{\mathbf{#1}}
\newcommand{\bs}[1]{\boldsymbol{#1}}
\newcommand{\myfigscale}{0.9}
\graphicspath{{figs/}}

\title{Parametric studies of composite material properties influence on dispersion curves of Lamb waves}

\author[a]{Pawel Kudela}
\author[a]{Piotr Fiborek}
\author[a]{Maciej Radzienski}
\author[a]{Tomasz Wandowski}
\affil[a]{Institute of Fluid-Flow Machinery, Polish Academy of Sciences, Fiszera 14 St, 80-231 Gdansk, Poland}

\authorinfo{Further author information: (Send correspondence to Pawel Kudela)\\Pawel Kudela: E-mail: pk@imp.gda.pl, Telephone: +48 58 5225 251}

% Option to view page numbers
\pagestyle{empty} % change to \pagestyle{plain} for page numbers   
\setcounter{page}{301} % Set start page numbering at e.g. 301
 
\begin{document} 
\maketitle

\begin{abstract}
\DIFdel{Typically,} material properties originate from destructive tensile tests and are used in computational models in the design and analysis process of structures. 
This approach is well-established in relation to isotropic homogeneous materials. 
However, if this approach is used for composite laminates, inaccuracies can arise that lead to vastly different stress distributions, strain rates, natural frequencies, and velocities of propagating elastic waves. 
In order to account for this problem, the alternative method is proposed, which utilizes Lamb wave propagation phenomenon and optimization techniques. 
Propagating Lamb waves are highly sensitive to changes in material parameters and are often used for structural health monitoring of structures.
In the proposed approach, the elastic constants, which are utilized to determine dispersion curves of Lamb waves, are optimized to achieve a good correlation between model predictions and experimental observations. 
In the first step of this concept, parametric studies have been carried out in which the influence of mass density, Young's modulus, Poisson's ratio of reinforcing fibers as well as a matrix of composite laminate and volume fraction of reinforcing fibers on dispersion curves of Lamb waves was investigated. 
The dispersion curves of Lamb waves were calculated by using the semi-analytical spectral element method considering the variability of properties of composite constituents. 
The resulting dispersion curves were also compared with experimental measurements of full wavefield data conducted by scanning laser Doppler vibrometer and processed by 3D Fourier transform. It allowed formulating fitness function which will be next used in genetic algorithm for optimization and identification of elastic constants.
\end{abstract}

% Include a list of keywords after the abstract 
\keywords{Lamb waves, dispersion curves, semi-analytical spectral element method, 
composite laminates, elastic constants }

\begin{equation}
\begin{aligned}
\begin{aligned}
\DIFdelmath{
c^2=a^2+b^2
%\label{eq:pitagoras}
}\\
\DIFaddmath{
c=\sqrt{a^2+b^2}
\label{eq:pitagoras}
}	
\end{aligned}
\end{aligned}
\end{equation}

\section{\DIFdel{Long title}}
\section[\DIFadd{short title}]{\DIFadd{Long title}}

\section{INTRODUCTION}
\label{sec:intro}  % \label{} allows reference to this section
In the last decades, high-performance composite materials such as carbon fiber reinforced plastics (CFRP) have been extensively used in various branches of industry.
It is because of great stiffness and strength to weight ratio in comparison to the metallic structure. 
However, the lay-up and orientation of reinforcing fibers introduce anisotropy and complexity in inspection regarding the detection and localization of defects.
Elastic constant values are often difficult to obtain, especially for anisotropic media.
These values are indispensable for the design of a structure which fulfills assumed 
requirements (strength, stiffness, vibration characteristics). 
Destructive methods have been utilized over the years for the characterization of isotropic materials. 
In contrast, the methods based on natural frequencies and ultrasonic waves can be implemented in a non-destructive way. 
However, the identification of elastic constants in composite materials is more complex than in the case of isotropic materials and usually requires the preparation of several specimens (cube cutting~\cite{Ditri1993}).

Ultrasonic techniques usually are based on measurements of the time of flight and corresponding velocities of bulk waves (longitudinal and shear)~\cite{Castellano2014}.
Bulk wave propagation is defined by the small wavelength compared to the thickness of the plate. 
In turn, small wavelengths correspond to high-frequency components which usually are strongly attenuated. 
Hence, there is an upper bound frequency limit.
Another problem arises from the assumption of the infinite thickness of the plate in the approximation of bulk waves. 
Neglecting the effect of plate boundaries causes errors in the time of flight estimation~\cite{Martens2017}. 

The above-mentioned limitations can be resolved by using an ultrasonic polar scan~\cite{Martens2019a}.
It enables determination of the elastic constants of composite material along with material damping characteristics.
Additionally, it has the potential for microscopic defects detection.
The disadvantage of such an approach is the fact that an inspected specimen must be immersed in water as this reduces the acoustic impedance mismatch at the
interfaces and allows sound to penetrate the sample more easily.

Another alternative method explored among researchers is a method that utilizes dispersion curves of Lamb waves.
It requires an excitation source in the form of a laser impulse or piezoelectric actuator permanently attached to the inspected specimen and a scanning laser Doppler measurement system.
Ong et al.~\cite{Ong2016} proposed a method in which experimental and numerical signals of Lamb waves acquired along lines corresponding to selected angles of propagation are used. 
Measurements are taken on the upper and bottom surface of the plate so that symmetric and antisymmetric modes can be separated.
Signals are processed by using a 2D Fourier transform in order to obtain dispersion curve patterns. 
Correlation between numerical and experimental dispersion curve patterns is considered in the objective function.
However, measurements are taken along a line which may cause a problem of the unwanted contribution of reflected waves from the boundaries of the plate.
Another downside is that access to both sides of the investigated specimen is required.
Therefore, to mitigate these downsides, we propose to utilize the full wavefield of propagating waves measured at only one side of the specimen.

The aim of this paper is the investigation of various parameters influencing the dispersion curves of Lamb waves.
This is a preliminary step that helps formulate objective function in the optimization problem leading to the determination of elastic constants of a composite laminate.

\section{SEMI-ANALYTICAL SPECTRAL ELEMENT METHOD}
\label{sec:sase}
Lamb wave dispersion phenomenon is related to wavenumber  \(k\) dependency on frequency \(f\). 
Lamb wave dispersion curves depend on elasticity constants of the material in which Lamb waves propagate.
The semi-analytical method was used for the calculation of dispersion curves.

The physical model of a plate-like waveguide is shown in Fig.~\ref{fig:layered_composite_SASE}.  
The waveguide is made of layered orthotropic material in which reinforcing fibers are at angle \(\theta\) in respect to \(z\) axis. 
The current mathematical model is a modification of the semi-analytical finite element (SAFE) method proposed in Ref.~\citenum{Bartoli2006}. 
The modification includes the application of spectral elements instead of finite elements through the thickness of a laminate, preserving wave equation in the propagation direction. 
Four-node spectral element is shown in Fig.~\ref{fig:layered_composite_SASE}. 
It has a non-uniform distribution of nodes and three degrees of freedom per node.

\begin{figure} [ht]
	\begin{center}
		\begin{tabular}{c} %% tabular useful for creating an array of images 
			\includegraphics{figure1.png}
		\end{tabular}
	\end{center}
	\caption[] 
	%>>>> use \label inside caption to get Fig. number with \ref{}
	{ \label{fig:layered_composite_SASE} 
		SASE model representation.}
\end{figure} 

The general form of wave equation which is used for calculation of dispersion curves has a form of eigenvalue problem:

\begin{equation}
\left[\matr{A} - \omega^2\matr{M} \right] \vect{U} =0,
\label{eq:eig_dispersion}
\end{equation}
where \(\omega\) is the angular frequency, \(\matr{M}\) is the mass matrix, \(\matr{U}\) is the nodal displacement vector, and the matrix \(\matr{A}\) can be defined as:
\begin{equation}
\begin{aligned}
\matr{A} & =  k^2\left(s^2 \,\matr{K}_{22} + c^2\, \matr{K}_{33} - c s\, \matr{K}_{23} - c s\, \matr{K}_{32}\right) \\
& + i k\, \matr{T}^T\left(-c\, \matr{K}_{13} - s\, \matr{K}_{21} + s\, \matr{K}_{12} + c\, \matr{K}_{31}\right) \matr{T} +\matr{K}_{11},
\end{aligned}
\label{eq:dispersion}
\end{equation}
where  \(s = \sin(\beta)\), \(c = \cos(\beta)\), \(i = \sqrt{-1}\), \(\beta\) is the angle 
of guided wave propagation and \(\matr{K}_{ij}\) are appropriate stiffness matrices~\cite{Taupin2011}. 
The transformation matrix \(\matr{T}\) is diagonal and it is introduced in order to eliminate imaginary elements from Eq.~(\ref{eq:dispersion}) (see Ref.~\citenum{Bartoli2006} for more details). 
It should be noted that the system of Eqs.~(\ref{eq:eig_dispersion}) explicitly depends on the angle \(\beta\). 
Predicting the anisotropic behavior of guided wave properties makes it necessary to loop over each direction considered.

\section{INDIRECT METHOD}
\label{sec:indirect}

The idea behind the indirect method is the determination of the following properties: matrix density  \(\rho_m\), fibers density \(\rho_f\), matrix Young's modulus \(E_m\), fibers Young's modulus along fibers \(E_{11f}\) and perpendicular to fibers \(E_{22f}=E_{33f}\), Poisson's ratio of the matrix \(\nu_m\), Poisson's ratio of the fibers \(\nu_f\) and volume fraction of reinforcing fibers \(V\). 
It was assumed that \(E_{22f} = 0.1\, E_{11f}\) thus the total number of independent parameters was reduced to~7. 
One parameter can be further eliminated if the overall density \(\rho\) of the composite is known. 
However, for the sake of completeness, the influence of all 7 parameters on dispersion curves was investigated here. 
The main motivation here is the low number of parameters which must be determined by the optimization method. 
Moreover, the material properties of composite constituents can be easily found in the literature. 
Therefore, can be used as a starting point in the optimization. 

It should be noted that the notation \(E_f = E_{11f}\) was used further in the text for simplicity. 
The effective elastic constants were calculated by using the rule of mixtures and homogenization techniques~\cite{Barbero2006a,Adumitroaie2012}.

\subsection{Material properties}
The material under investigation is made out of 8 layers of plain-weave reinforced composite. 
Initial values of constants characterizing the composite material constituents (epoxy resin and carbon fibers) are given in Tab.~\ref{tab:matprop}.

\begin{table}[ht]
		\renewcommand{\arraystretch}{1.3}
		%\centering \footnotesize
		\caption{Initial values of constants characterizing the composite material 
			constituents.}
		
		\label{tab:matprop}
		\begin{center}	
			\begin{tabular}{ccccccc} 
				\toprule
				\multicolumn{3}{c}{\textbf{Matrix} }	& \multicolumn{3}{c}{\textbf{Fibres} } & 
				\textbf{Volume fraction}	 \\ 
				\midrule
				\(\rho_m\) & \(E_m\) & \(\nu_m\)  & \(\rho_f\) & \(E_f\) & \(\nu_f\) & \(V\)\\
				kg/m\textsuperscript{3} &GPa& --  & kg/m\textsuperscript{3}  & GPa& -- & \%\\ 
				\cmidrule(lr){1-3} \cmidrule(lr){4-6} \cmidrule(lr){7-7}
				1250 &3.43& 0.35& 1900 & 240 & 0.2 & 50\\
				\bottomrule 
			\end{tabular} 
		\end{center}
\end{table}


The assumed thickness of the composite laminate was 3.9~mm and the mass density 1522.4~kg/m\textsuperscript{3}. 
The~parameters describing the geometry of a plain weave textile-reinforced composite 
are given in Tab.~\ref{tab:weave_geo}. 
\begin{table}[ht]
	\renewcommand{\arraystretch}{1.3}
	%\centering \footnotesize
	\caption{The geometry of a plain weave textile-reinforced composite [mm].}
	\label{tab:weave_geo}
	\begin{center}
		\begin{tabular}{cccccc} 
			\toprule
			\multicolumn{4}{c}{\textbf{width} }	& \multicolumn{2}{c}{\textbf{thickness} }  \\ 
			\cmidrule(lr){1-4} \cmidrule(lr){5-6} 
			fill & warp & fill gap& warp gap& fill & warp\\
			\(a_f\) &\(a_w\)& \(g_f\)  & \(g_w\)  & \(h_f\)& \(h_w\) \\ 
			\cmidrule(lr){1-2} \cmidrule(lr){3-4} \cmidrule(lr){5-6}
			1.92 &2.0& 0.05& 0.05 & 0.121875 & 0.121875 \\
			\bottomrule 
		\end{tabular} 
	\end{center}
\end{table}

\cleardoublepage
\subsection{Results of parametric studies}
The influence of each parameter on dispersion curves was studied separately. 
The variability range of each parameter was assumed as \(\pm\)20\% with respect to initial values.
The results of parametric studies are presented in Figs.~\ref{fig:rhom}--\ref{fig:vol}.
The dispersion curves corresponding to four Lamb wave modes are in the form \(k(f)\) where \(f=\omega/(2 \pi)\) is the frequency measured in hertz. 
Black curves are calculated for the initial values of material properties given in Tab.~\ref{tab:matprop}, red curves represent changes in dispersion curves caused by the increase of these parameters whereas blue curves represent changes in dispersion curves due to decreasing values of these parameters. 
There are 11 solutions covering the range of \(\pm\)20\% for each investigated parameter (5 red curves, 5 blue curves and 1 black curve for each propagating Lamb wave mode).

The influence of matrix density on dispersion curves for the selected angle \(\beta\) is shown in Fig.~\ref{fig:rhom}. 
The parameter has a moderate influence on each mode of propagating waves.
A similar trend can be observed in the case of fiber density as shown in Fig.~\ref{fig:rhof}.
It should be noted that the increase of matrix density and fibers density causes an increase of wavenumber values across the whole frequency range.

\newtcolorbox{mybox}{colback=red!5!white,
	colframe=red!75!black}
\begin{mybox}
	This is my own box.
\end{mybox}


\begin{center}
\begin{tikzpicture}
\node[anchor=south west,inner sep=0] (image) at (0,0) {
		\DIFdelfig{\includegraphics{figure2a.png}}
};
\begin{scope}[x={(image.south east)},y={(image.north west)}]
\draw[mygrey,line width=1 mm] (0, 1)--(1, 0);
\draw[mygrey,line width=1 mm] (0, 0)--(1, 1);
\end{scope}
\end{tikzpicture}
\end{center}
\begin{figure} [ht]
	\begin{center}
		\begin{tabular}{cc} %% tabular useful for creating an array of images 
		
				\begin{tikzpicture}
				\node[anchor=south west,inner sep=0] (image) at (0,0) {
					\DIFdelfig{\includegraphics{figure2a.png}}
				};
				\begin{scope}[x={(image.south east)},y={(image.north west)}]
				\draw[mygrey,line width=1 mm] (0, 1)--(1, 0);
				\draw[mygrey,line width=1 mm] (0, 0)--(1, 1);
				\end{scope}
				\end{tikzpicture}
			
			\DIFaddfig{
			\includegraphics[scale=\myfigscale]{figure2b.png}
			}
		\end{tabular}
	\end{center}
	\caption[] 
	%>>>> use \label inside caption to get Fig. number with \ref{}
	{ \label{fig:rhom} 
		The influence of matrix density on the dispersion curves.}
\end{figure} 

\begin{figure} [ht]
	
	\begin{center}
		\begin{tabular}{cc} %% tabular useful for creating an array of images 
			\includegraphics[scale=\myfigscale,draft]{figure3a.png}
			
			\DIFaddfig{\includegraphics[scale=\myfigscale]{figure3b.png}}
		\end{tabular}
	\end{center}

	\caption[] 
	%>>>> use \label inside caption to get Fig. number with \ref{}
	{ \label{fig:rhof} 
		The influence of fiber density on the dispersion curves.}
\end{figure} 

The influence of Young's modulus of the matrix on dispersion curves for the selected angle 
\(\beta\) is shown in Fig.~\ref{fig:em}. 
Large changes in most dispersion curves can be observed. 
But, changes in certain dispersion curves at frequencies up to about 300~kHz are small (these dispersion curves correspond to S0 modes of Lamb waves).
The influence of Young's modulus of fibers on dispersion curves for the selected angle 
\(\beta\) is shown in Fig.~\ref{fig:ef}. 
In this case small to moderate changes in dispersion curves depending on guided wave mode can be observed. 

Moreover, it is worthy to notice, that the effect of Young's modulus on dispersion curves is opposite to the mass density. 
Namely, an increase of  Young's modulus causes a decrease of wavenumber values.
\begin{figure} [ht]
	\begin{center}
		\begin{tabular}{cc} %% tabular useful for creating an array of images 
			\includegraphics[scale=\myfigscale]{figure4a.png}
			\includegraphics[scale=\myfigscale]{figure4b.png}
		\end{tabular}
	\end{center}
	\caption[] 
	%>>>> use \label inside caption to get Fig. number with \ref{}
	{\DIFdel{ \label{fig:em}} 
		The influence of Young's modulus of matrix on dispersion curves. \label{fig:em2}}
\end{figure} 

\begin{figure} [ht]
	\begin{center}
		\begin{tabular}{cc} %% tabular useful for creating an array of images 
			\includegraphics[scale=\myfigscale]{figure5a.png}
			\includegraphics[scale=\myfigscale]{figure5b.png}
		\end{tabular}
	\end{center}
	\caption[] 
	%>>>> use \label inside caption to get Fig. number with \ref{}
	{ \label{fig:ef} 
		The influence of Young's modulus of fibers on dispersion curves.}
\end{figure} 
%\newpage
\pagebreak
The influence of  Poisson's ratio of the matrix on the dispersion curves for the selected angle \(\beta\) is shown in Fig.~\ref{fig:nim}. 
Small changes in dispersion curves can be observed.
Even smaller effects on dispersion curves can be observed for the case of Poisson's ratio of fibers (see Fig.~\ref{fig:nif}).
Depending on the frequency the and propagating mode an increase of Poisson's ratio can increase or decease wavenumber values.
\begin{figure} [ht]
	\begin{center}
		\begin{tabular}{cc} %% tabular useful for creating an array of images 
			\includegraphics[scale=\myfigscale]{figure6a.png}
			\includegraphics[scale=\myfigscale]{figure6b.png}
		\end{tabular}
	\end{center}
	\caption[] 
	%>>>> use \label inside caption to get Fig. number with \ref{}
	{ \label{fig:nim} 
		The influence of Poisson's ratio of matrix on dispersion curves.}
\end{figure} 
\begin{figure} [ht]
	\begin{center}
		\begin{tabular}{cc} %% tabular useful for creating an array of images 
			\includegraphics[scale=\myfigscale]{figure7a.png}
			\includegraphics[scale=\myfigscale]{figure7b.png}
		\end{tabular}
	\end{center}
	\caption[] 
	%>>>> use \label inside caption to get Fig. number with \ref{}
	{ \label{fig:nif} 
		The influence of Poisson's ratio of fibers on dispersion curves.}
\end{figure} 

The influence of volume fraction of reinforcing fibers on dispersion curves for the selected angle \(\beta\) is shown in Fig.~\ref{fig:vol}. 
It is the most essential parameter among analyzed properties because it has the greatest influence on dispersion curves of guided waves.
\begin{figure} [ht]
	\begin{center}
		\begin{tabular}{cc} %% tabular useful for creating an array of images 
			\includegraphics[scale=\myfigscale]{figure8a.png}
			\includegraphics[scale=\myfigscale]{figure8b.png}
		\end{tabular}
	\end{center}
	\caption[] 
	%>>>> use \label inside caption to get Fig. number with \ref{}
	{ \label{fig:vol} 
		The influence of volume fraction of reinforcing fibers on dispersion curves.}
\end{figure} 

It should be underlined that in all cases considered the changes in dispersion curves are higher at higher frequencies. 
Therefore, it is expected that a higher accuracy of elastic constants determination can be achieved if dispersion curves are captured experimentally at higher frequencies. Moreover, as expected, the dependence of dispersion curves on the angle of propagation is significant. 
Hence, this fact should be considered during the construction of an objective function.

\section{DIRECT METHOD}
\label{sec:direct}

The idea behind the direct method is the determination of the following properties:
\(C_{11}\), \(C_{12}\), \(C_{13}\) , \(C_{22}\), \(C_{23}\), \(C_{33}\), \(C_{44}\) , \(C_{55}\), \(C_{66}\). These elastic constants are describing orthotropic elastic material and are used in stress-strain constitutive equations. 
Therefore there are more constants characterizing material properties to be determined than in case of the indirect method (9 versus 6 if we assume that the mass density is known).

In the case of direct approach, upper and lower bounds of variables were set as  
\(\pm\)20\% in respect to initial values given in Tab.~\ref{tab:Ctensor_initial}. 
The initial values were selected so that to match behavior of plain-weave type composite as described in the indirect method.

\begin{table}[h!]
	\renewcommand{\arraystretch}{1.3}
	%\centering \footnotesize
	\caption{Initial values of elastic constants used in parametric studies (direct method). 
	Units: [GPa]}
		\label{tab:Ctensor_initial}
	\begin{center}
		\begin{tabular}{ccccccccc} 
			\toprule
			\(C_{11}\) & \(C_{12}\) & \(C_{13}\)  & \(C_{22}\) & \(C_{23}\) & \(C_{33}\) & 
			\(C_{44}\)  & \(C_{55}\) & \(C_{66}\) \\
			\midrule
			50 &5& 5&  50 & 5 & 9 & 3 & 3 & 3\\
			\bottomrule 
		\end{tabular} 
	\end{center}
\end{table}

\subsection{Results of parametric studies}
As previously, the influence of each parameter on dispersion curves was studied separately. 
The results of parametric studies are presented in a similar manner in Figs.~\ref{fig:C11}--\ref{fig:C66}.
Immediately, it can be noticed that the behavior of dispersion curves is completely different than in the case of the indirect method. 
The influence of \(C_{ij}\) elastic constants is more localized, meaning that certain elastic constants affect certain modes of Lamb waves. 
In particular, for example \(C_{11}\) constant at the propagation angle 0\(^{\circ}\) affects most significantly the S0 mode of Lamb waves (Fig.~\ref{fig:C11} ) whereas \(C_{44}\) constant at propagation angle 0\(^{\circ}\) affects A1 Lamb wave mode (Fig.~\ref{fig:C44}).

Moreover, the influence of elastic constants on dispersion curves is angle-dependent. 
It means that  for example \(C_{11}\) constant at propagation angle 0\(^{\circ}\) affects dispersion curves more than in case of the propagation angle 90\(^{\circ}\). 
An increase or decrease of  \(C_{11}\) constants has no effect on dispersion curves at  the propagation angle 90\(^{\circ}\). 

Additionally, certain changes in dispersion curves are observed at a particular frequency range only. 
For example in case of  \(C_{33}\) constant, S0 mode of Lamb waves is significantly affected in the frequency range 250--500 kHz (see Fig.~\ref{fig:C33}).

It should be also underlined that the increase/decrease of elastic constant values represented by red/blue curves is specific to a particular elastic constant. 
The increase of elastic constants casing decrease values of wavenumbers can be observed for the case of: \(C_{ij}, i=j\), whereas opposite behavior can be observed for the case of  \(C_{ij}, i\ne j\).      
\begin{figure} [ht]
	\begin{center}
		\begin{tabular}{cc} %% tabular useful for creating an array of images 
			\includegraphics[scale=\myfigscale]{figure9a.png}
			\includegraphics[scale=\myfigscale]{figure9b.png}\\
			\includegraphics[scale=\myfigscale]{figure9c.png}
			\includegraphics[scale=\myfigscale]{figure9d.png}
		\end{tabular}
	\end{center}
	\caption[] 
	%>>>> use \label inside caption to get Fig. number with \ref{}
	{ \label{fig:C11} 
		The influence of \(C_{11}\) elastic constant on the dispersion curves.}
\end{figure} 

The influence of \(C_{12}\) elastic constant on the dispersion curves is least pronounced (see Fig.~\ref{fig:C12}). 
It might lead to difficulty in the proper determination of this constant through optimization method based on dispersion curves.

\(C_{13}\) constants affect dispersion curves at frequency range about 250--500 kHz (see Fig.~\ref{fig:C13}).
Changes in dispersion curves at angle 90\(^{\circ}\) are not visible.

In the case of \(C_{22}\) constants (Fig.~\ref{fig:C22}), the largest changes in dispersion curves are for the propagation angle  30\(^{\circ}\). 

The reader can find other peculiarities in the remaining Figs.~\ref{fig:C23}--\ref{fig:C66}.
\begin{figure} [ht]
	\begin{center}
		\begin{tabular}{cc} %% tabular useful for creating an array of images 
			\includegraphics[scale=\myfigscale]{figure10a.png}
			\includegraphics[scale=\myfigscale]{figure10b.png}\\
			\includegraphics[scale=\myfigscale]{figure10c.png}
			\includegraphics[scale=\myfigscale]{figure10d.png}
		\end{tabular}
	\end{center}
	\caption[] 
	%>>>> use \label inside caption to get Fig. number with \ref{}
	{ \label{fig:C12} 
		The influence of \(C_{12}\) elastic constant on the dispersion curves.}
\end{figure} 

\begin{figure} [ht]
	\begin{center}
		\begin{tabular}{cc} %% tabular useful for creating an array of images 
			\includegraphics[scale=\myfigscale]{figure11a.png}
			\includegraphics[scale=\myfigscale]{figure11b.png}\\
			\includegraphics[scale=\myfigscale]{figure11c.png}
			\includegraphics[scale=\myfigscale]{figure11d.png}
		\end{tabular}
	\end{center}
	\caption[] 
	%>>>> use \label inside caption to get Fig. number with \ref{}
	{ \label{fig:C13} 
		The influence of \(C_{13}\) elastic constant on the dispersion curves.}
\end{figure} 

\begin{figure} [ht]
	\begin{center}
		\begin{tabular}{cc} %% tabular useful for creating an array of images 
			\includegraphics[scale=\myfigscale]{figure12a.png}
			\includegraphics[scale=\myfigscale]{figure12b.png}\\
			\includegraphics[scale=\myfigscale]{figure12c.png}
			\includegraphics[scale=\myfigscale]{figure12d.png}
		\end{tabular}
	\end{center}
	\caption[] 
	%>>>> use \label inside caption to get Fig. number with \ref{}
	{ \label{fig:C22} 
		The influence of \(C_{22}\) elastic constant on the dispersion curves.}
\end{figure} 

\begin{figure} [ht]
	\begin{center}
		\begin{tabular}{cc} %% tabular useful for creating an array of images 
			\includegraphics[scale=\myfigscale]{figure13a.png}
			\includegraphics[scale=\myfigscale]{figure13b.png}\\
			\includegraphics[scale=\myfigscale]{figure13c.png}
			\includegraphics[scale=\myfigscale]{figure13d.png}
		\end{tabular}
	\end{center}
	\caption[] 
	%>>>> use \label inside caption to get Fig. number with \ref{}
	{ \label{fig:C23} 
		The influence of \(C_{23}\) elastic constant on the dispersion curves.}
\end{figure} 

\begin{figure} [ht]
	\begin{center}
		\begin{tabular}{cc} %% tabular useful for creating an array of images 
			\includegraphics[scale=\myfigscale]{figure14a.png}
			\includegraphics[scale=\myfigscale]{figure14b.png}\\
			\includegraphics[scale=\myfigscale]{figure14c.png}
			\includegraphics[scale=\myfigscale]{figure14d.png}
		\end{tabular}
	\end{center}
	\caption[] 
	%>>>> use \label inside caption to get Fig. number with \ref{}
	{ \label{fig:C33} 
		The influence of \(C_{33}\) elastic constant on the dispersion curves.}
\end{figure} 

\begin{figure} [ht]
	\begin{center}
		\begin{tabular}{cc} %% tabular useful for creating an array of images 
			\includegraphics[scale=\myfigscale]{figure15a.png}
			\includegraphics[scale=\myfigscale]{figure15b.png}\\
			\includegraphics[scale=\myfigscale]{figure15c.png}
			\includegraphics[scale=\myfigscale]{figure15d.png}
		\end{tabular}
	\end{center}
	\caption[] 
	%>>>> use \label inside caption to get Fig. number with \ref{}
	{ \label{fig:C44} 
		The influence of \(C_{44}\) elastic constant on the dispersion curves.}
\end{figure} 

\begin{figure} [ht]
	\begin{center}
		\begin{tabular}{cc} %% tabular useful for creating an array of images 
			\includegraphics[scale=\myfigscale]{figure16a.png}
			\includegraphics[scale=\myfigscale]{figure16b.png}\\
			\includegraphics[scale=\myfigscale]{figure16c.png}
			\includegraphics[scale=\myfigscale]{figure16d.png}
		\end{tabular}
	\end{center}
	\caption[] 
	%>>>> use \label inside caption to get Fig. number with \ref{}
	{ \label{fig:C55} 
		The influence of \(C_{55}\) elastic constant on the dispersion curves.}
\end{figure} 

\begin{figure} [ht]
	\begin{center}
		\begin{tabular}{cc} %% tabular useful for creating an array of images 
			\includegraphics[scale=\myfigscale]{figure17a.png}
			\includegraphics[scale=\myfigscale]{figure17b.png}\\
			\includegraphics[scale=\myfigscale]{figure17c.png}
			\includegraphics[scale=\myfigscale]{figure17d.png}
		\end{tabular}
	\end{center}
	\caption[] 
	%>>>> use \label inside caption to get Fig. number with \ref{}
	{ \label{fig:C66} 
		The influence of \(C_{66}\) elastic constant on the dispersion curves.}
\end{figure} 
\clearpage
\section{CONCLUSIONS}
The influence of mechanical properties of materials on the dispersion curves of Lamb waves has been studied.
The semi-analytical spectral element method was used for the calculation of dispersion curves. 
It enabled us to discern the increase/decrease of wavenumber values of particular Lamb wave modes at selected frequencies.
Two methods were investigated: indirect and direct.
In the indirect method constituents of composite materials were treated as input variables which then were used for calculation of elastic constants.
In the direct method, direct changes of elastic constants in the range of \(\pm\)20\% with respect to initial values were studied.

It might seem that from the perspective of an optimization problem, it is easier to solve it if there are fewer parameters for determination. 
Thus, the indirect method is better in this regard.
 However, it has been observed that parameters such as Young's modulus and volume fraction of reinforcing fibers compete with each other in the overlapping range of wavenumber values. 
 Moreover, dispersion curves are affected in a similar way irrespective of the propagation angle.
 This can lead to ambiguities of optimal solutions. Actually, it is highly likely that a completely different set of indirect parameters can lead to the same dispersion curves.
 
 The parametric studies of the influence of elastic constants (\(C\) tensor) on the dispersion curves in the direct method show much more localized changes.
 Namely, there is a strict correlation between elastic constant and changes in wavenumber values of particular Lamb wave mode, frequency range or angle of propagation. 
 Therefore, it can be deduced that such an approach will lead to less ambiguous solutions in an optimization problem with a greater chance of finding a global minimum.
 
 The next step of our research studies is formulation of the objective function which will be used in the optimization process by using a genetic algorithm. 
 We are planning to transform calculated dispersion curves into an image.
 Such an image will be used as a filtering mask on a dispersion curves obtained from the laser vibrometer experimental measurements.
 The sum of filtered values will be a measure of the match between measured and calculated dispersion curves.
 
\appendix    %>>>> this command starts appendixes


\acknowledgments % equivalent to \section*{ACKNOWLEDGMENTS}       
 
The research was funded by the Polish National Science Center under grant agreement 
no 2018/29/B/ST8/00045. 

% References
\bibliographystyle{spiebib} % makes bibtex use spiebib.bst
\bibliography{SPIE2020} % 

\end{document} 
