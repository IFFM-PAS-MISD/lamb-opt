\documentclass[]{spie}  %>>> use for US letter paper
%\documentclass[a4paper]{spie}  %>>> use this instead for A4 paper
%\documentclass[nocompress]{spie}  %>>> to avoid compression of citations

\renewcommand{\baselinestretch}{1.0} % Change to 1.65 for double spacing
 
\usepackage{amsmath,amsfonts,amssymb}
\usepackage{graphicx}
\usepackage[colorlinks=true, allcolors=blue]{hyperref}
% my packages and commands
\usepackage{booktabs} % for nice tables
\graphicspath{{figs/}}

\title{Parametric studies of composite material properties influence on dispersion 
curves of Lamb waves}

\author[a]{Pawel Kudela}
\author[a]{Piotr Fiborek}
\author[a]{Maciej Radzienski}
\author[a]{Tomasz Wandowski}
\affil[a]{Institute of Fluid-Flow Machinery, Polish Academy of Sciences, Fiszera 14 St, 
80-231 Gdansk, Poland}

\authorinfo{Further author information: (Send correspondence to Pawel 
Kudela)\\Pawel Kudela: E-mail: pk@imp.gda.pl, Telephone: +48 58 5225 251}

% Option to view page numbers
\pagestyle{empty} % change to \pagestyle{plain} for page numbers   
\setcounter{page}{301} % Set start page numbering at e.g. 301
 
\begin{document} 
\maketitle

\begin{abstract}
Typically, material properties originate from destructive tensile tests and are used in 
computational models in the design and analysis process of structures. 
This approach is well-established in relation to isotropic homogeneous materials. 
However, if this approach is used for composite laminates, inaccuracies can arise that 
leads to vastly different stress distributions, strain rates, natural frequencies, and 
velocities of propagating elastic waves. In order to account for this problem, the 
alternative method is proposed, which utilizes Lamb wave propagation phenomenon 
and optimization techniques. 
Propagating Lamb waves are highly sensitive to changes in material parameters and 
are often used for structural health monitoring of structures. In the proposed approach, 
the elastic constants, which are utilized to determine dispersion curves of Lamb waves, 
are optimized to achieve a good correlation between model predictions and 
experimental observations. 
In the first step of this concept, parametric studies have been carried out in which the 
influence of mass density, Young’s modulus, Poisson’s ratio of reinforcing fibers as 
well as a matrix of composite laminate and volume fraction of reinforcing fibers on 
dispersion curves of Lamb waves was investigated. 
The dispersion curves of Lamb waves were calculated by using the semi-analytical 
spectral element method considering the variability of properties of composite 
constituents. 
The resulting dispersion curves were also compared with experimental measurements 
of full wavefield data conducted by scanning laser Doppler vibrometer and processed 
by 3D Fourier transform. It allowed formulating fitness function which will be next used 
in genetic algorithm for optimization and identification of elastic constants.
\end{abstract}

% Include a list of keywords after the abstract 
\keywords{Lamb waves, dispersion curves, semi-analytical spectral element method, 
composite laminates, elastic constants }

\section{INTRODUCTION}
\label{sec:intro}  % \label{} allows reference to this section

Elastic constant values are often difficult to obtain, especially for anisotropic media.
These values are indispensable for the design of a structure which fulfills assumed 
requirements (strength, stiffness, vibration characteristics). 
Several methods for identification of elastic constants of isotropic materials have been 
utilized over the years, namely: destructive testing and strain gauge or static 
displacement recordings. 
Other methods such as methods based on natural frequencies and measurements of 
velocities of bulk waves (longitudinal and shear) can be implemented in non-destructive 
way. 
However, identification of elastic constants in composite materials is more complex 
than in case of isotropic materials and usually requires preparation of several 
specimens (cube cutting).
The elastic constants for composite materials along with material damping 
characteristics can be obtained by using ultrasonic polar scan.
Another alternative is a method which utilizes dispersion 
curves of Lamb waves.

References to equations include the equation number in parentheses, for example, 
Equation~(\ref{eq:rho}) shows ...''or `Combining Eqs.~(2) and (3), we obtain...

 Using a tilde in the LaTeX source file between two characters avoids unwanted line breaks.

Authors are encouraged to follow the principles of sound technical writing, as described 
in Refs.~\citenum{Alred03} and \citenum{Perelman97}, for example.
please see \citenum{Lees-Miller-LaTeX-course-1}. 

\section{SEMI-ANALYTICAL SPECTRAL ELEMENT METHOD}
\label{sec:sase}

\begin{figure} [ht]
	\begin{center}
		\begin{tabular}{c} %% tabular useful for creating an array of images 
			\includegraphics{figure1.png}
		\end{tabular}
	\end{center}
	\caption[] 
	%>>>> use \label inside caption to get Fig. number with \ref{}
	{ \label{fig:sase} 
		SASE model representation.}
\end{figure} 

\section{INDIRECT METHOD}
\label{sec:indirect}

The idea behind the indirect method is the determination of the following properties:  
matrix  density \(\rho_m\), fibers density \(\rho_f\), matrix Young's modulus \(E_m\), fibers 
Young's  modulus \(E_f\), Poisson's ratio of the matrix \(\nu_m\), Poisson's ratio of the 
fibers\(\nu_f\)  and  volume fraction of reinforcing fibers\(V\).  
In reality the fiber density can be calculated based on  the rule of mixture as:
\begin{equation}
\rho_f = \frac{\rho - \rho_m (1-V)}{V},
\label{eq:rho}
\end{equation}
where $V$ is the volume fraction of reinforcing fibers. Hence, only 6~variables needs to be 
determined. 
However, for the sake of completeness the influence of all 7 parameters on dispersion curves 
was investigated.

\subsection{Material properties}
\begin{table}[ht]
	\renewcommand{\arraystretch}{1.3}
	%\centering \footnotesize
	\caption{Initial values of constants characterizing the composite material 
		constituents.}
	\label{tab:matprop}
	\begin{center}	
		\begin{tabular}{ccccccc} 
			\toprule
			\multicolumn{3}{c}{\textbf{Matrix} }	& \multicolumn{3}{c}{\textbf{Fibres} } & 
			\textbf{Volume fraction}	 \\ 
			\midrule
			\(\rho_m\) & \(E_m\) & \(\nu_m\)  & \(\rho_f\) & \(E_f\) & \(\nu_f\) & \(V\)\\
			kg/m\textsuperscript{3} &GPa& --  & kg/m\textsuperscript{3}  & GPa& -- & \%\\ 
			\cmidrule(lr){1-3} \cmidrule(lr){4-6} \cmidrule(lr){7-7}
			1250 &3.43& 0.35& 1900 & 240 & 0.2 & 50\\
			\bottomrule 
		\end{tabular} 
	\end{center}
\end{table}

\subsection{Homogenization}
The assumed thickness of composite laminate was 3.9~mm. 

The total weight of the specimen was 8550 g, hence, based on the volume of the 
specimen, the density is about~1522.4~kg/m\textsuperscript{3}.

The parameters describing the geometry of a plain weave textile reinforced composite 
are given in Table~\ref{tab:weave_geo}. 
\begin{table}[ht]
	\renewcommand{\arraystretch}{1.3}
	%\centering \footnotesize
	\caption{Geometry of a plain weave textile reinforced composite [mm].}
	\label{tab:weave_geo}
	\begin{center}
		\begin{tabular}{cccccc} 
			\toprule
			\multicolumn{4}{c}{\textbf{width} }	& \multicolumn{2}{c}{\textbf{thickness} }  \\ 
			\cmidrule(lr){1-4} \cmidrule(lr){5-6} 
			fill & warp & fill gap& warp gap& fill & warp\\
			\(a_f\) &\(a_w\)& \(g_f\)  & \(g_w\)  & \(h_f\)& \(h_w\) \\ 
			\cmidrule(lr){1-2} \cmidrule(lr){3-4} \cmidrule(lr){5-6}
			1.92 &2.0& 0.05& 0.05 & 0.121875 & 0.121875 \\
			\bottomrule 
		\end{tabular} 
	\end{center}
\end{table}

\subsection{Results of parametric studies}

\begin{figure} [ht]
	\begin{center}
		\begin{tabular}{cc} %% tabular useful for creating an array of images 
			\includegraphics{figure2a.png}
			\includegraphics{figure2b.png}
		\end{tabular}
	\end{center}
	\caption[] 
	%>>>> use \label inside caption to get Fig. number with \ref{}
	{ \label{fig:rhom} 
		The influence of matrix density on the dispersion curves.}
\end{figure} 

\begin{figure} [ht]
	\begin{center}
		\begin{tabular}{cc} %% tabular useful for creating an array of images 
			\includegraphics{figure3a.png}
			\includegraphics{figure3b.png}
		\end{tabular}
	\end{center}
	\caption[] 
	%>>>> use \label inside caption to get Fig. number with \ref{}
	{ \label{fig:rhof} 
		The influence of fiber density on the dispersion curves.}
\end{figure} 

\begin{figure} [ht]
	\begin{center}
		\begin{tabular}{cc} %% tabular useful for creating an array of images 
			\includegraphics{figure4a.png}
			\includegraphics{figure4b.png}
		\end{tabular}
	\end{center}
	\caption[] 
	%>>>> use \label inside caption to get Fig. number with \ref{}
	{ \label{fig:em} 
		The influence of Young's modulus of matrix on dispersion curves.}
\end{figure} 

\begin{figure} [ht]
	\begin{center}
		\begin{tabular}{cc} %% tabular useful for creating an array of images 
			\includegraphics{figure5a.png}
			\includegraphics{figure5b.png}
		\end{tabular}
	\end{center}
	\caption[] 
	%>>>> use \label inside caption to get Fig. number with \ref{}
	{ \label{fig:ef} 
		The influence of Young's modulus of fibers on dispersion curves.}
\end{figure} 

\begin{figure} [ht]
	\begin{center}
		\begin{tabular}{cc} %% tabular useful for creating an array of images 
			\includegraphics{figure6a.png}
			\includegraphics{figure6b.png}
		\end{tabular}
	\end{center}
	\caption[] 
	%>>>> use \label inside caption to get Fig. number with \ref{}
	{ \label{fig:nim} 
		The influence of Poisson's ratio of matrix on dispersion curves.}
\end{figure} 

\begin{figure} [ht]
	\begin{center}
		\begin{tabular}{cc} %% tabular useful for creating an array of images 
			\includegraphics{figure7a.png}
			\includegraphics{figure7b.png}
		\end{tabular}
	\end{center}
	\caption[] 
	%>>>> use \label inside caption to get Fig. number with \ref{}
	{ \label{fig:nif} 
		The influence of Poisson's ratio of fibers on dispersion curves.}
\end{figure} 

\begin{figure} [ht]
	\begin{center}
		\begin{tabular}{cc} %% tabular useful for creating an array of images 
			\includegraphics{figure8a.png}
			\includegraphics{figure8b.png}
		\end{tabular}
	\end{center}
	\caption[] 
	%>>>> use \label inside caption to get Fig. number with \ref{}
	{ \label{fig:vol} 
		The influence of volume fraction of reinforcing fibers on dispersion curves.}
\end{figure} 

\clearpage
\section{DIRECT METHOD}
\label{sec:direct}
The idea behind the indirect method is the determination of the following properties:
\(C_{11}\), \(C_{12}\), \(C_{13}\) , \(C_{22}\), \(C_{23}\), \(C_{33}\), \(C_{44}\) , \(C_{55}\), 
\(C_{66}\).

In case of direct approach, upper and lower bounds of variables were set as  
\(\pm\)50\% in respect to initial values given in Table~\ref{tab:Ctensor_initial}.
\begin{table}[h!]
	\renewcommand{\arraystretch}{1.3}
	%\centering \footnotesize
	\caption{Initial values of elastic constants used in parametric studies (direct method). 
	Units: [GPa]}
		\label{tab:Ctensor_initial}
	\begin{center}
		\begin{tabular}{ccccccccc} 
			\toprule
			\(C_{11}\) & \(C_{12}\) & \(C_{13}\)  & \(C_{22}\) & \(C_{23}\) & \(C_{33}\) & 
			\(C_{44}\)  & \(C_{55}\) & \(C_{66}\) \\
			\midrule
			50 &5& 5&  50 & 5 & 9 & 3 & 3 & 3\\
			\bottomrule 
		\end{tabular} 
	\end{center}
\end{table}

\subsection{Results of parametric studies}


\begin{figure} [ht]
	\begin{center}
		\begin{tabular}{cc} %% tabular useful for creating an array of images 
			\includegraphics{figure9a.png}
			\includegraphics{figure9b.png}\\
			\includegraphics{figure9c.png}
			\includegraphics{figure9d.png}
		\end{tabular}
	\end{center}
	\caption[] 
	%>>>> use \label inside caption to get Fig. number with \ref{}
	{ \label{fig:C11} 
		The influence of \(C_{11}\) elastic constant on the dispersion curves.}
\end{figure} 

\begin{figure} [ht]
	\begin{center}
		\begin{tabular}{cc} %% tabular useful for creating an array of images 
			\includegraphics{figure10a.png}
			\includegraphics{figure10b.png}\\
			\includegraphics{figure10c.png}
			\includegraphics{figure10d.png}
		\end{tabular}
	\end{center}
	\caption[] 
	%>>>> use \label inside caption to get Fig. number with \ref{}
	{ \label{fig:C12} 
		The influence of \(C_{12}\) elastic constant on the dispersion curves.}
\end{figure} 

\begin{figure} [ht]
	\begin{center}
		\begin{tabular}{cc} %% tabular useful for creating an array of images 
			\includegraphics{figure11a.png}
			\includegraphics{figure11b.png}\\
			\includegraphics{figure11c.png}
			\includegraphics{figure11d.png}
		\end{tabular}
	\end{center}
	\caption[] 
	%>>>> use \label inside caption to get Fig. number with \ref{}
	{ \label{fig:C13} 
		The influence of \(C_{13}\) elastic constant on the dispersion curves.}
\end{figure} 

\begin{figure} [ht]
	\begin{center}
		\begin{tabular}{cc} %% tabular useful for creating an array of images 
			\includegraphics{figure12a.png}
			\includegraphics{figure12b.png}\\
			\includegraphics{figure12c.png}
			\includegraphics{figure12d.png}
		\end{tabular}
	\end{center}
	\caption[] 
	%>>>> use \label inside caption to get Fig. number with \ref{}
	{ \label{fig:C22} 
		The influence of \(C_{22}\) elastic constant on the dispersion curves.}
\end{figure} 

\begin{figure} [ht]
	\begin{center}
		\begin{tabular}{cc} %% tabular useful for creating an array of images 
			\includegraphics{figure13a.png}
			\includegraphics{figure13b.png}\\
			\includegraphics{figure13c.png}
			\includegraphics{figure13d.png}
		\end{tabular}
	\end{center}
	\caption[] 
	%>>>> use \label inside caption to get Fig. number with \ref{}
	{ \label{fig:C23} 
		The influence of \(C_{23}\) elastic constant on the dispersion curves.}
\end{figure} 

\begin{figure} [ht]
	\begin{center}
		\begin{tabular}{cc} %% tabular useful for creating an array of images 
			\includegraphics{figure14a.png}
			\includegraphics{figure14b.png}\\
			\includegraphics{figure14c.png}
			\includegraphics{figure14d.png}
		\end{tabular}
	\end{center}
	\caption[] 
	%>>>> use \label inside caption to get Fig. number with \ref{}
	{ \label{fig:C33} 
		The influence of \(C_{33}\) elastic constant on the dispersion curves.}
\end{figure} 

\begin{figure} [ht]
	\begin{center}
		\begin{tabular}{cc} %% tabular useful for creating an array of images 
			\includegraphics{figure15a.png}
			\includegraphics{figure15b.png}\\
			\includegraphics{figure15c.png}
			\includegraphics{figure15d.png}
		\end{tabular}
	\end{center}
	\caption[] 
	%>>>> use \label inside caption to get Fig. number with \ref{}
	{ \label{fig:C44} 
		The influence of \(C_{44}\) elastic constant on the dispersion curves.}
\end{figure} 

\begin{figure} [ht]
	\begin{center}
		\begin{tabular}{cc} %% tabular useful for creating an array of images 
			\includegraphics{figure16a.png}
			\includegraphics{figure16b.png}\\
			\includegraphics{figure16c.png}
			\includegraphics{figure16d.png}
		\end{tabular}
	\end{center}
	\caption[] 
	%>>>> use \label inside caption to get Fig. number with \ref{}
	{ \label{fig:C55} 
		The influence of \(C_{55}\) elastic constant on the dispersion curves.}
\end{figure} 

\begin{figure} [ht]
	\begin{center}
		\begin{tabular}{cc} %% tabular useful for creating an array of images 
			\includegraphics{figure17a.png}
			\includegraphics{figure17b.png}\\
			\includegraphics{figure17c.png}
			\includegraphics{figure17d.png}
		\end{tabular}
	\end{center}
	\caption[] 
	%>>>> use \label inside caption to get Fig. number with \ref{}
	{ \label{fig:C66} 
		The influence of \(C_{66}\) elastic constant on the dispersion curves.}
\end{figure} 
\clearpage
\section{DISCUSSION}

  
\appendix    %>>>> this command starts appendixes


\acknowledgments % equivalent to \section*{ACKNOWLEDGMENTS}       
 
The research was funded by the Polish National Science Center under grant agreement 
no 2018/29/B/ST8/00045. 

% References
\bibliographystyle{spiebib} % makes bibtex use spiebib.bst
%\bibliography{report} % bibliography data in report.bib

\end{document} 
