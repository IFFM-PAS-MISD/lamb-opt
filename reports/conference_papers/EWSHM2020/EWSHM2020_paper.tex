% This is samplepaper.tex, a sample chapter demonstrating the
% LLNCS macro package for Springer Computer Science proceedings;
% Version 2.20 of 2017/10/04
%
\documentclass[runningheads]{llncs}
%
\usepackage[top=5cm, bottom=5.6cm, left=4.5cm, right=4.2cm]{geometry}
\usepackage{graphicx}
\usepackage{array}
\usepackage{caption}
\usepackage{subcaption}  % fpr multiple figure
\usepackage{gensymb}     % for degree symbole
\usepackage{booktabs}    % for nice tables
\usepackage{csvsimple}   % for csv read

\newcolumntype{P}[1]{>{\centering\arraybackslash}p{#1}}
% Used for displaying a sample figure. If possible, figure files should
% be included in EPS format.
%
% If you use the hyperref package, please uncomment the following line
% to display URLs in blue roman font according to Springer's eBook style:
% \renewcommand\UrlFont{\color{blue}\rmfamily}

\makeatletter
\renewcommand\paragraph{\@startsection{paragraph}{4}{\z@}%
                                    {3.25ex \@plus1ex \@minus.2ex}%
                                    {-1em}%
                                    {\normalfont\normalsize\bfseries}}
\makeatother

\begin{document}
%
\title{Strategies for identification of elastic constants in highly anisotropic materials using Lamb waves}
%
\titlerunning{Strategies for identification of elastic constants using Lamb waves}
% If the paper title is too long for the running head, you can set
% an abbreviated paper title here
%
\author{Maciej Radzieński\inst{1}\orcidID{0000-0003-2571-0052} \and
Paweł Kudela\inst{1}\orcidID{0000-0002-5130-6443}  \and 
Tomasz Wandowski\inst{1}\orcidID{0000-0003-1414-2969} 
\and Wiesław Ostachowicz\inst{1}\orcidID{0000-0002-8061-8614}
}
%
\authorrunning{M. Radzieński et al.}
% First names are abbreviated in the running head.
% If there are more than two authors, 'et al.' is used.
%
\institute{Institute of Fluid-Flow Machinery Polish Academy of Sciences
\email{maciej.radzienski@imp.gda.pl}}
%
\maketitle              % typeset the header of the contribution
%
%\begin{abstract}
\paragraph{Abstract.}
Information about exact material properties may be of great importance in many areas where CAD/CAE software is used. It is also a key component of properly operating model-based SHM systems. Unfortunately, composite laminates producers are not providing sufficient and/or precise enough materials data sheets to meet such requirements. This is the reason why material properties identification techniques are attracting considerable interest.
\\
This paper presents a new, non-destructive elastic constants identification technique based on Lamb wave phenomenon. Experimental dispersion curves are obtained by 3D Fourier transform of full wavefield time response registered in a tested sample by scanning laser Doppler vibrometer. Numerical dispersion curves, generated by a semi-analytical element model, are optimized to match experimental dispersion curves. By minimizing the discrepancies between two sets of data, the elastic constants are identified. 
\\
Various strategies are tested, where optimization techniques are used to fit dispersion curves in the wavenumber-frequency domain for chosen propagation angles or the wavenumber-angle domain for chosen frequencies. Both indirect and direct approaches were used where composite constituents and C-tensor components where optimized correspondingly.




\keywords{Lamb waves \and Elastic constants identification \and Optimization techniques.}
%\end{abstract}
%
\\[2em]
%
\section{Introduction}

Ultrasonic methods for determination of elastic constants of composite laminates has been recently enhanced by the utilization of the ultrasonic polar scan method~\cite{Martens2019a}.

An alternative approach is based on signals of propagating Lamb waves.
Ong et al.~\cite{Ong2016} proposed a method in which experimental and numerical signals acquired along lines corresponding to selected angles of propagation are used. 
Measurements are taken on the upper and bottom surface of the plate so that symmetric and antisymmetric modes can be separated.
Signals are processed by using 2D Fourier transform in order to obtain dispersion curve patterns. 
Correlation between numerical and experimental dispersion curve patterns is considered in the objective function.
However, measurements taken along a line my cause a problems of contribution of reflected waves from boundaries of the plate.
Therefore, we propose to utilize full wavefield of propagating waves in construction of the objective function.

\section{Semi-analitycal Finite Element Model}
\section{Experimental set-up}

Lamb waves were excited by 10 mm round piezoelectric disk (PZT) attached to the specimens back surface.  Chirp signal with the frequency range 0-500 kHz lasting 200 $\mu$s was generated every 10 ms and applied to PZT element throught signal amplifier. 

Speciment central area of 455 x 455 mm was measured in 499 x 499 points using Scanning Laser Vibrometer. In every measurment point 1024 time samples were registered with sampling frequency 1.28 MHz. The response in every point was repeated 20 times and avaraged. 

\begin{figure}
	\begin{center}
	\includegraphics{figs/Setup.png}
	\end{center}
	\caption{Experimental set-up.}
	\label{fig1}
\end{figure}


\section{Results}

\begin{figure}
	\begin{center}
		\includegraphics{figs/X_Y_t_slice_499x499p_chp200_x30_6Vpp_250Hz_100mmsv_small_uni.png}
	\end{center}
	\caption{Space-space-time data.}
	\label{fig2}
\end{figure}

\begin{figure}
	\begin{center}
		\includegraphics{figs/kx-ky-f_slices2__499x499p_chp200_x30_6Vpp_250Hz_100mmsv_small_uni.png}
	\end{center}
	\caption{Wavenumber-Wavenumber-Frequency data sliced at 100 and 400 kHz.}
	\label{fig3}
\end{figure}

\begin{figure}
	\begin{center}
		\includegraphics{figs/KXKYF_slices_2499x499p_chp200_x30_6Vpp_250Hz_100mmsv_small_uni.png}
	\end{center}
	\caption{Wavenumber-Wavenumber-Frequency data sliced at 45 and 90 degrees of wave propagation angle.}
	\label{fig4}
\end{figure}

\subsection{Wavenumber-Frequency for 7 wave propagation angles}

\begin{figure}[ht]
	\begin{subfigure}{.5\textwidth}
		\centering
		% include first image
		\includegraphics[width=.8\linewidth]{figs/ga_unidirectional_C_tensor_known_mass_angle_0_dispersion_curves_test_case_25_small.png}  
	%\caption{}
	\end{subfigure}
	\begin{subfigure}{.5\textwidth}
		\centering
		% include second image
		\includegraphics[width=.8\linewidth]{figs/ga_unidirectional_C_tensor_known_mass_angle_15_dispersion_curves_test_case_25_small.png}  
	%\caption{}
	\end{subfigure}
	\newline
	\begin{subfigure}{.5\textwidth}
		\centering
		% include first image
		\includegraphics[width=.8\linewidth]{figs/ga_unidirectional_C_tensor_known_mass_angle_30_dispersion_curves_test_case_25_small.png}  
	%\caption{}
	\end{subfigure}
	\begin{subfigure}{.5\textwidth}
		\centering
		% include second image
		\includegraphics[width=.8\linewidth]{figs/ga_unidirectional_C_tensor_known_mass_angle_45_dispersion_curves_test_case_25_small.png}  
	%\caption{}
	\end{subfigure}
	\caption{Fitted dispertion curve on experimental data for chosen wave propagation angles}
	\label{fig7}
\end{figure}

 \begin{table}[h]
	\renewcommand{\arraystretch}{1.3}
	\centering \footnotesize
	\caption{Initial values of constants characterising the composite material constituents.}
	%\begin{tabular}{@{}ccccccc@{}} % remove spaces from vertical lines
	\begin{tabular}{ccccccc} 
		%\hline
		\toprule
		\multicolumn{3}{c}{\textbf{Matrix} }	& \multicolumn{3}{c}{\textbf{Fibres} } & \textbf{Volume fraction}	 \\ 
		%	\hline \hline
		\midrule
		$\rho_m$ & $E_m$ & $\nu_m$  & $\rho_f$ & $E_f$ & $\nu_f$ & $V$\\
		% \cmidrule(lr){1-3} \cmidrule(lr){4-6} \cmidrule(lr){7-7}
		%\hline
		kg/m\textsuperscript{3} &GPa& --  & kg/m\textsuperscript{3}  & GPa& -- & \%\\ 
		%\hline
		%\midrule
		\cmidrule(lr){1-3} \cmidrule(lr){4-6} \cmidrule(lr){7-7}
		1250 &3.43& 0.35& 1900 & 240 & 0.2 & 50\\
		%\hline 
		\bottomrule 
	\end{tabular} 
	\label{tab:matprop}
\end{table}


\begin{table}[h]
	\renewcommand{\arraystretch}{1.3}
	\centering \footnotesize
	\caption{GA optimisation results based on statistics (mean $\mu$ and standard deviation $\sigma$) of 50 GA runs; Units of  elastic constants: [GPa].}	
	\begin{tabular}{crrrrrr} \toprule
		&\multicolumn{3}{c}{\textbf{indirect method}} \\
		%\midrule
		\cmidrule(lr){2-4} 
		&Best & $\mu$ & $\sigma$ \\
		\cmidrule(lr){2-4} 
		\csvreader[table head=\toprule ,
		late after line=\\ ]%
		{results_direct_50.csv}{Row=\constants,Cdbest=\cdbest,Cdmean=\cdmean,Cdstd=\cdstd}%
		{\constants & \cdbest & \cdmean & \cdstd}%	
		\bottomrule
	\end{tabular}	
	\label{tab:csv_results}
\end{table}

\subsection{Wavenumber-Wavenumber for 7 frequencies}

test

\begin{figure}[ht]
	\begin{subfigure}{.5\textwidth}
		\centering
		% include first image
		\includegraphics[width=.8\linewidth]{figs/ga_unidirectional_C_tensor_known_mass_kx_ky_frequency_2_dispersion_surf_test_case_1_small.png}  
		\label{fig:sub-first}
	\end{subfigure}
	\begin{subfigure}{.5\textwidth}
		\centering
		% include second image
		\includegraphics[width=.8\linewidth]{figs/ga_unidirectional_C_tensor_known_mass_kx_ky_frequency_3_dispersion_surf_test_case_1_small.png}  
		\label{fig:sub-second}
	\end{subfigure}
	\newline
	\begin{subfigure}{.5\textwidth}
		\centering
		% include first image
		\includegraphics[width=.8\linewidth]{figs/ga_unidirectional_C_tensor_known_mass_kx_ky_frequency_4_dispersion_surf_test_case_1_small.png}  
		\label{fig:sub-first}
	\end{subfigure}
	\begin{subfigure}{.5\textwidth}
		\centering
		% include second image
		\includegraphics[width=.8\linewidth]{figs/ga_unidirectional_C_tensor_known_mass_kx_ky_frequency_5_dispersion_surf_test_case_1_small.png}  
		\label{fig:sub-second}
	\end{subfigure}
	\caption{Fitted dispertion curve on experimental data for chosen frequencies}
	\label{fig7}
\end{figure}

\section{Conclutions}


%
% ---- Bibliography ----
%
% BibTeX users should specify bibliography style 'splncs04'.
% References will then be sorted and formatted in the correct style.
%
 \bibliographystyle{splncs04}
 \bibliography{EWSHM2020}
%

\end{document}
